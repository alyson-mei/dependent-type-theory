\documentclass[12pt]{article}
% Page setup
\usepackage[margin=1in]{geometry}
\usepackage{parskip}
\usepackage{titling}
\setlength{\droptitle}{-6em} 

% Math packages
\usepackage{amsmath, amsthm, amssymb, amsfonts, stmaryrd}
\usepackage{mathtools}
\makeatletter
\g@addto@macro\th@plain{\normalfont} % remove italics from theorems (plain style)
\makeatother

% Commutative diagrams
\usepackage{tikz-cd}
\usepackage{proof}

\usepackage{hyperref}

\hypersetup{
    colorlinks=true,
    linkcolor=blue,
    urlcolor=cyan,
    citecolor=green,
    allcolors=blue
}

% Theorem environments
\newtheorem{theorem}{Theorem}[section]
\newtheorem{lemma}[theorem]{Lemma}
\newtheorem{proposition}[theorem]{Proposition}
\newtheorem{corollary}[theorem]{Corollary}
\newtheorem{fact}[theorem]{Fact}
\newtheorem{remark}[theorem]{Remark}

\theoremstyle{definition}
\newtheorem{definition}[theorem]{Definition}
\newtheorem{example}[theorem]{Example}
\newtheorem{exercise}[theorem]{Exercise}

% Useful commands for category theory


\newcommand{\mc}{\mathcal}
\newcommand{\ini}{\operatorname{ini}}
\newcommand{\ter}{\operatorname{ter}}
\newcommand{\cod}{\operatorname{cod}}
\newcommand{\dom}{\operatorname{dom}}

\newcommand{\List}{\operatorname{List}}
\newcommand{\lop}{\operatorname{[ \ ]}}
\newcommand{\len}{\operatorname{len}}
\newcommand{\type}{\operatorname{type}}

\newcommand{\Exc}{\operatorname{Exception}}

\newcommand{\Kl}{\operatorname{Kl}}
\newcommand{\plus}{\operatorname{+}}

\newcommand{\Cat}{\mathcal{C}at}
\newcommand{\Vect}{\mathrm{Vect}_\mathbf{k}}
\newcommand{\Hom}{\operatorname{Hom}}
\newcommand{\Nat}{\operatorname{Nat}}
\newcommand{\id}{\mathrm{id}}
\newcommand{\op}{\mathrm{op}}
\newcommand{\obj}{\operatorname{obj}}
\newcommand{\arr}{\operatorname{arr}}
\newcommand{\adj}{\operatorname{adj}}
\newcommand{\coadj}{\operatorname{coadj}}
\newcommand{\colim}{\operatorname*{colim}}

\newcommand{\abort}{\operatorname{abort}}
\newcommand{\fst}{\operatorname{fst}}
\newcommand{\snd}{\operatorname{snd}}
\newcommand{\inl}{\operatorname{inl}}
\newcommand{\inr}{\operatorname{inr}}
\newcommand{\case}{\operatorname{case}}


% Arrows
\newcommand{\ra}{\rightarrow}
\newcommand{\from}{\leftarrow}
\newcommand{\To}{\Rightarrow}
\newcommand{\xto}{\xrightarrow}
\newcommand{\xfrom}{\xleftarrow}

\renewcommand{\labelitemi}{--} % Level 1: Dashes
\renewcommand{\labelitemii}{$\circ$} % Level 2: Circles
\renewcommand{\labelitemiii}{$\bullet$} % Level 3: Bullets (default)
\newcommand{\andd}{\mathbin{\&}}
\newcommand{\dash}{\textrm{-}}

\title{Notes on Dependent Type Theory}
\author{Alyson Mei}
\date{\today}

\begin{document}

\maketitle

These notes are based on \href{https://www.youtube.com/watch?v=y-lX1mx5_i0&list=PLt7hcIEdZLAnbUxKaG7XIynEWrozOBtXU}{video recordings} of R. Harper's lectures.

\section{Introduction}

Boolean algebra is associated with classical logic, Heyting algebra -- with intuitionistic logic.

\begin{definition}[Boolean algebra]
A \emph{Boolean algebra} can be defined as a complemented distributive lattice:
\begin{itemize}
    \item[$\bullet$] Pre-order:
    $$ x \leq x, \qquad x \leq y \andd y \leq z \to x \leq z; $$
    \item[$\bullet$] Has finite meets and joins:
    $$x \leq 1, \qquad z \leq x \andd z \leq y \to z \leq (x \wedge y), \qquad x \wedge y \leq x, \qquad x \wedge y \leq y;$$
    $$0 \leq x, \qquad x \leq z \andd y \leq z \to x \vee y \leq z, \qquad x \leq x \vee y, \qquad y \leq x \vee y;$$
    \item[$\bullet$] Has complements:
    $$1 \leq \bar{x} \vee x, \qquad \bar{x} \wedge x \leq 0;$$  
    \item[$\bullet$] Distributive:
    $$x \wedge (y \vee z) \equiv (x \wedge y) \vee (x \wedge z), \qquad
      x \vee (y \wedge z) \equiv (x \vee y) \wedge (x \vee z).$$
\end{itemize}

Additionally, the exponential is defined as
$$y^x := \bar{x} \vee y.$$
\end{definition}

\begin{definition}[Heyting algebra]
A \emph{Heyting algebra} is defined as a lattice with exponentials:
\begin{itemize}
    \item[$\bullet$] Pre-order;
    \item[$\bullet$] Has finite meets and joins;
    \item[$\bullet$] Has exponentials:
    $$y^x \wedge x \leq y, \qquad z \wedge x \leq y \to z \leq y^x.$$
\end{itemize}
\end{definition}

\begin{exercise}
\hfill
\begin{enumerate}
    \item ``Yoneda lemma'': $x \leq y$ iff $\forall z \ (z \leq x \to z \leq y)$;
    \item Every Heyting algebra is distributive.
\end{enumerate}
\end{exercise}

Quotes:
\begin{itemize}
    \item ``Boolean algebra (closed world) = Heyting algebra (open world) with complements'';
    \item ``Classical logic is a logic with complete information''.
\end{itemize}

The definitions provide us with standard rules:
\begin{itemize}
    \item Weakening:
    $$x \leq x \vee y \qquad (x \wedge y \leq x);$$
    \item Contraction:
    $$x \leq x \wedge x;$$
    \item Exchange:
    $$x \wedge y \equiv y \wedge x.$$
\end{itemize}

\textbf{Relation to classical logic}:
\begin{itemize}
    \item Sequent $\Gamma \vdash A$, where $\Gamma = A_1, \dots, A_n$, corresponds to:
    $$A_1 \wedge \dots \wedge A_n \leq A;$$
    \item Rules for $\wedge$:
    \begin{center}
        \begin{tabular}{ccc}
        $\displaystyle \infer{\Gamma \vdash A \wedge B}{\Gamma \vdash A & \Gamma \vdash B}$ 
        &
        $\displaystyle \infer{\Gamma, A \wedge B \vdash A}{}$
        &
        $\displaystyle \infer{\Gamma, A \wedge B \vdash B}{}$
        \text{;}
        \end{tabular}
    \end{center}
    \item ...and so on.
\end{itemize}

\begin{definition}[Lindenbaum algebra]
A \emph{Lindenbaum algebra} is defined as the algebra of equivalence classes of a given theory:
\begin{itemize}
    \item[$\bullet$] $[A] = \lbrace B \mid B \equiv A \rbrace;$ 
    \item[$\bullet$] $[A] \wedge [B] := [A \wedge B];$
    \item ... and so on.
\end{itemize}
\end{definition}

\begin{theorem}[Soundness and Completeness for Intuitionistic Propositional Logic]
Let $\Gamma$ be a context and $A$ a formula. Then
\[
\Gamma \vdash A
\quad \text{iff} \quad
\forall H \;
\bigl(
\llbracket \Gamma \rrbracket_H \leq \llbracket A \rrbracket_H
\bigr),
\]
where $H$ ranges over all Heyting algebras, and
$\llbracket - \rrbracket_H$ denotes the interpretation of formulas as elements
of $H$ under a valuation of propositional variables.
\end{theorem}

\newpage

\section{Simple Type Theory}

\begin{definition}[Simple type theory]
    We define the \emph{simple type theory} as follows:

\begin{itemize}
    \item[$\bullet$] Unit 1:
        \begin{center}
        \begin{tabular}{ccc}
        $\displaystyle \infer[1\dash F]{\Gamma \vdash 1 \ \type}{}$ 
        &
        $\displaystyle \infer[1 \dash I]{\Gamma \vdash \langle \rangle : 1}{}$
        &
        $\displaystyle \infer[\textrm{(no 1-E)}]{}{}$
        \text{;}
        \end{tabular}
        \end{center}
        
    \item[$\bullet$] Product $\times$:
        \begin{center}
        \begin{tabular}{ccc}
        $\displaystyle \infer[\times \dash F]{\Gamma \vdash A \times B \ \type}{\Gamma \vdash A \ \type & \Gamma \vdash  B \ \type}$ 
        &
        $\displaystyle \infer[\times \dash I]{\Gamma \vdash \langle M, N \rangle : A \times B}{\Gamma \vdash M: A & \Gamma \vdash N : B}$
        &
        $\displaystyle \infer[\times\text{-}E]{
          \begin{array}{c}
            \Gamma \vdash \fst(M) : A \\
             \Gamma \vdash \snd(M) : B
          \end{array}
        }{
          \Gamma \vdash M : A \times B
        }$
        \text{;}
        \end{tabular}
        \end{center}

    \item[$\bullet$] Exponential $\to$:
        \begin{center}
        \begin{tabular}{ccc}
            $\displaystyle \infer[\to \dash F]{\Gamma \vdash A \to B \ \type}{\Gamma \vdash A \ \type & \Gamma \vdash  B \ \type}$ 
            &
            $\displaystyle \infer[\to \dash I]{\Gamma \vdash \lambda x. M: A \to B}{\Gamma, x: A \vdash M:B}$
            &
        \end{tabular}

        \vspace{2mm}

        \begin{tabular}{c}
            $\displaystyle \infer[\to \dash E]{\Gamma \vdash M(N):B}{\Gamma \vdash M: A \to B & \Gamma \vdash N: A}$
            \text{;}
        \end{tabular}
        \end{center}

    \item[$\bullet$] Void 0:
        \begin{center}
            \begin{tabular}{ccc}
            $\displaystyle \infer[0\dash F]{\Gamma \vdash 0 \ \type}{}$ 
            &
            $\displaystyle \infer[\textrm{(no 0-I)}]{}{}$
            &
            $\displaystyle \infer[0\dash E]{\Gamma \vdash \abort(M) : A}{\Gamma \vdash M: 0}$ 
            \text{;}
            \end{tabular}
        \end{center}
        
        \item[$\bullet$] Coproduct $+$:
        \begin{center}
        \begin{tabular}{ccc}
        
        $\displaystyle 
        \infer[+\dash F]
          {\Gamma \vdash A + B \ \type}
          {\Gamma \vdash A \ \type 
           & 
           \Gamma \vdash B \ \type}
        $
        
        &
        
        $\displaystyle
        \infer[+\dash I_1]
          {\Gamma \vdash \inl(M) : A + B}
          {\Gamma \vdash M : A}
        $
        
        &
        
        $\displaystyle
        \infer[+\dash I_2]
          {\Gamma \vdash \inr(N) : A + B}
          {\Gamma \vdash N : B}
        $
        \text{,}
        \\[3mm]
        
        \multicolumn{3}{c}{
        $\displaystyle
        \infer[+\text{-}E]
          {\Gamma, z: A + B \vdash
            \case (x.N, y. P)(z): C}
          {
            \Gamma, x: A \vdash N: C 
            &
            \Gamma, y:B \vdash P : C
          }
        $
        \text{.}
        }
        
        \end{tabular}
        \end{center}

\end{itemize}
\end{definition}

\begin{definition}[$\beta$ and $\eta$ equivalences]
    \hfill
    \begin{itemize}
        \item Unit 1:
        $$(\beta) \ \text{none} \qquad (\eta) \ \Gamma \vdash \langle \rangle \equiv M : 1;$$
        \item Product $\times$:
            $$
            (\beta)\;
            \begin{aligned}
            &\Gamma \vdash \fst(\langle M, N \rangle) \equiv M : A \\
            &\Gamma \vdash \snd(\langle M, N \rangle) \equiv N : B
            \end{aligned}
            \qquad
            (\eta) \ \Gamma \vdash \langle \fst M, \snd M\rangle \equiv M : A \times B;
            $$
        \item Exponential $\to$:
        $$(\beta) \ \Gamma \vdash (\lambda x. M)(N) \equiv [N / x] M: B \qquad (\eta) \ \Gamma \vdash (\lambda x. M(x)) \equiv M: A \to B;$$

        \item Zero $0$:
        $$
        (\beta) \ \text{none} \qquad
        (\eta)\; \Gamma, z: 0 \vdash R \equiv \abort(M) : C;
        $$
        
         \item Coproduct $+$:
        $$
        (\beta)\;
        \begin{aligned}
        &\case(x.M; y.N)(\inl(P)) \equiv [P/x]M : C \\
        &\case(x.M; y.N)(\inr(Q)) \equiv [Q/y]N : C
        \end{aligned}
        \text{\ ,}
        $$
        $$
        \infer[(\eta)]{\Gamma, z: A + B \vdash R \equiv \case(x.M; y.N)(z)}{\Gamma \vdash [\inl(P)/z]R \equiv [P/x]M & \Gamma \vdash [\inr(Q)/z]R \equiv [Q/y]N}
        \text{.}
        $$
                

        
    \end{itemize}
\end{definition}

\end{document}
